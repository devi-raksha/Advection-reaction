\documentclass[11pt]{article}
\usepackage{amsmath,amssymb,amsthm,mathtools}
\usepackage[a4paper,margin=1in]{geometry}
\usepackage{bm}

\title{Discontinuous Galerkin variational formulation for a 1D embedded Burgers equation}
\author{}
\date{}

\newcommand{\R}{\mathbb{R}}
\newcommand{\avg}[1]{\left\{\!\!\left\{#1\right\}\!\!\right\}}
\newcommand{\jump}[1]{\left[\!\left[#1\right]\!\right]}
\newcommand{\dt}{\Delta t}
\newcommand{\Uh}{u_h}
\newcommand{\Vh}{V_h}
\newcommand{\Th}{\mathcal{T}_h}
\newcommand{\Fh}{\mathcal{F}_h}
\newcommand{\Kh}{K}
\newcommand{\Gh}{\mathcal{G}_h}
\newcommand{\norm}[1]{\left\lVert #1\right\rVert}

\begin{document}
\maketitle

\section{Model problem and geometry}
We solve the nonconservative form of the (scalar) Burgers equation in 1D,
\begin{equation}
  \partial_t u + \nabla_s \cdot \!\left(\bm b\frac{u^2}{2}\right) = f
  \quad\text{on } \Gamma\times(0,T],\qquad
  u(\,\cdot\,,0)=u_0,
  \label{eq:model}
\end{equation}
where $\Gamma$ is a straight 1D segment embedded in $\R^{\text{spacedim}}$ and
$s$ is arclength along $\Gamma$. In the code $\Gamma$ is represented by
line cells with endpoints $\bm x_0,\bm x_1$ and unit \emph{tangent}
\[
\bm b := \frac{\bm x_1-\bm x_0}{\lVert \bm x_1-\bm x_0\rVert}.
\]
For any face with outward unit normal $\bm n$, the tangential--normal coupling used by the numerical flux is the scalar
\[
\beta := \bm b\cdot \bm n\in[-1,1].
\]
Manufactured data use
$u_{\text{ex}}(x,t)=e^{-t}\sin(\pi x)$ and $f=u_t+\partial_s(u^2/2)$.

\section{Discrete space and notation}
Let $\Th$ be a partition of $\Gamma$ into line cells $\Kh$. For degree $k\ge0$,
\[
\Vh := \{ v\in L^2(\Gamma)\;:\; v|_{\Kh}\in\mathbb P_k(\Kh)\ \forall \Kh\},
\]
realized in the code by \texttt{FE\_DGQ$<1,\,$spacedim$>$}. For $v^\pm$ the traces on an interior face $F\in\Fh^\circ$ with a fixed normal $\bm n$,
\[
\avg{v}=\tfrac12\,(v^-+v^+),\qquad
\jump{v}=v^- - v^+ .
\]
On boundary faces $F\subset\partial\Gamma$ we set $\avg{v}=v$, $\jump{v}=v$ and use the outward normal.

The $L^2$ mass bilinear form is
\[
m(w,v) := \sum_{\Kh\in\Th}\int_{\Kh} w\,v\,\mathrm ds ,
\]
assembled in \texttt{assemble\_mass\_matrix()}.

\section{Time discretization and linearization}
Let $t^{n+1}=t^n+\dt$. The code uses backward Euler in time with a Picard linearization of the nonlinear flux around the current iterate.
Denote by $u^{n}$ the solution at $t^n$ and by $u^{n+1,(m)}$ the $m$-th Picard iterate for $t^{n+1}$.
The linearization of $\partial_s(u^2/2)$ is
\[
\partial_s\!\left(\frac{u^2}{2}\right)
\approx \partial_s\!\left(\frac{u^{n+1, (m)}\,u^{n+1,(m+1)}}{2}\right)
= \frac{u^{n+1, (m)}}{2}\,\partial_s u^{n+1,(m+1)} 
\]
A relaxation step $u^{n+1,(m+1)} \leftarrow \omega\,u^{n+1,(m+1)} +
(1-\omega)\,u^{n+1,(m)}$ is applied with user parameter $\omega\in(0,1]$.

At each Picard step the linear system is
\begin{equation}
\left(\frac{1}{\dt}M + A(u^{n+1,(m)})\right) u^{n+1,(m+1)} \;=\; \frac{1}{\dt}M u^n + F(t^{n+1}),
\label{eq:time_system}
\end{equation}
where $M$ is the mass matrix and $A(u^{n+1,(m)})$ is the DG advection operator using $(u^{n+1,(m)})$ as coefficient. This is built in \texttt{assemble\_system()} and solved either by UMFPACK or GMRES.

\section{Semi-implicit DG bilinear and linear forms}
Let $w$ denote the trial function and $v$ the test function. The operator $A(z)$ is composed of:
\paragraph{Cell term.}
For each $\Kh$,
\begin{equation}
a_{\text{cell}}(z; w,v)
:= - \int_{\Kh} \frac{z}{2}\, (\bm b\cdot\nabla v)\, w \,\mathrm ds .
\label{eq:cell}
\end{equation}
This matches the code line
\(\texttt{- 0.5 * u\_old\_q * b\_vec * value(j) * gradient(i)}\),
i.e.\ the trial value multiplies the test gradient.

\paragraph{Interior face flux.}
For each interior face $F\in\Fh^\circ$ with normal $\bm n$ and $\beta=\bm b\cdot\bm n$,
\begin{equation}
a_{\text{int}}(z; w,v)
:= \sum_{F\in\Fh^\circ}\int_F
\Big(
\frac\beta2\,\avg{z}\,\jump{v}\,\avg{w}
+ \theta\,\left|\frac\beta2\,\avg{z}\right|\,\jump{v}\,\jump{w}
\Big)\,\mathrm d\ell .
\label{eq:interior}
\end{equation}
Here $\theta>0$ is a user penalty. In the code
\(\avg{z}\) is implemented as \(\tfrac12\,(u_L^{n+1,(m)}+u_R^{n+1,(m)})\); the resulting integrand equals \eqref{eq:interior}. The structure is intentionally one-sided: the test function appears with $\jump{v}$; the trial function appears with $\avg{w}$ and with $\jump{w}$ in the penalty. This yields an upwinded, consistent and stable semi-implicit flux for linear advection speed \(z\).

\paragraph{Boundary faces.}
Let $g$ denote boundary data at $t^{n+1}$. Let $u_{\text{ext}}^n$ be the exterior state obtained from the exact solution at $t^n$ (code: \texttt{u\_ext\_old}). With $\beta=\bm b\cdot\bm n$,
\begin{align}
a_{\partial}(u^n; w,v) &:= \sum_{F\subset\partial\Gamma}
\int_F \beta\,\frac{u_{\text{ext}}^n}{2}\, v\, w \,\mathrm d\ell,
&& \text{if }\beta\,\frac{u_{\text{ext}}^n}{2}>0 \quad \text{(outflow)}, \label{eq:bdry-matrix}
\\
\ell_{\partial}(u^n; v) &:= \sum_{F\subset\partial\Gamma}
\int_F \left(-\beta\,\frac{u_{\text{ext}}^n}{2}\right)\, g\, v \,\mathrm d\ell,
&& \text{if }\beta\,\frac{u_{\text{ext}}^n}{2}\le 0 \quad \text{(inflow)}. \label{eq:bdry-rhs}
\end{align}
Thus outflow contributes a matrix term, inflow imposes data via the right-hand side, following standard upwind logic with characteristic speed $\beta\,u^n/2$.

\paragraph{Source term.}
At $t^{n+1}$,
\begin{equation}
\ell_{\text{vol}}(t^{n+1}; v) := \sum_{\Kh\in\Th}\int_\Kh f(\cdot,t^{n+1})\, v \,\mathrm ds .
\end{equation}

\paragraph{Global forms.}
Combine
\begin{equation}
A(u^n; w,v) := a_{\text{cell}} + a_{\text{int}} + a_{\partial},\qquad
F(t^{n+1}; v) := \ell_{\text{vol}} + \ell_{\partial}.
\end{equation}
The assembled linear system at each Picard step is exactly \eqref{eq:time_system}.

\section{Mass matrix and discrete norms}
The code assembles $M$ with element-wise $\int_\Kh \phi_i\phi_j$.
It also evaluates the discrete $L^2$ norm via $u^\top M u$ and compares against
$\norm{u_{\text{ex}}(\cdot,t)}_{L^2(0,1)} = \sqrt{1/2}\,e^{-t}$.

\section{Remarks on consistency and stability}
\begin{itemize}
  \item Consistency: if $u$ is smooth and single-valued, $\jump{u}=0$ and the face terms reduce to the continuous flux with coefficient $u^n$; \eqref{eq:cell} follows from integration by parts of $\int \tfrac{u^n}{2}\partial_s w\, v$ using the DG test space.
  \item Upwinding: the sign of $\beta\,u^n/2$ selects inflow/outflow at the boundary and, on interior faces, the term $\beta\,\avg{u^n}\,\jump{v}\,\avg{w}$ acts as a semi-implicit upwind flux.
  \item Penalty: $\theta\,|\beta\,\avg{u^n}|\,\jump{v}\,\jump{w}$ controls interelement jumps and stabilizes the transport operator for variable coefficient $u^n$. The parameter \texttt{theta} is user-set.
  \item Time stepping: backward Euler is unconditionally stable on the linearized problem. Nonlinearity is handled by Picard with optional relaxation $\omega$.
\end{itemize}

\section{Algorithm summary (as implemented)}
\begin{enumerate}
  \item Build $M$ once: $M_{ij}=\sum_\Kh\int_\Kh \phi_j\,\phi_i$.
  \item For each time step:
    \begin{enumerate}
      \item Initialize Picard iterate $u^{n+1,(0)}:=u^{n}$.
      \item Repeat until convergence or max iterations:
        \begin{enumerate}
          \item Assemble $A(u^{n,(m)})$ using the current coefficient field $u^{n,(m)}$ (in code, the vector named \texttt{solution} at assembly time) via \eqref{eq:cell}--\eqref{eq:bdry-rhs}.
          \item Form $\frac{1}{\dt}M + A(u^{n,(m)})$ and right-hand side $\frac{1}{\dt}Mu^{n} + F(t^{n+1})$.
          \item Solve for $u^{n+1,(m+1)}$.
          \item Relax: $u^{n+1,(m+1)} \leftarrow \omega\,u^{n+1,(m+1)} + (1-\omega)\,u^{n+1,(m)}$.
        \end{enumerate}
      \item Set $u^{n+1}\gets u^{n+1,(m_\star)}$ and continue.
    \end{enumerate}
\end{enumerate}

\section{Implementation mapping}
\begin{itemize}
  \item Cell form \eqref{eq:cell}: \texttt{copy.cell\_matrix(i,j) += -0.5 * u\_old\_q * (b\_vec) * value(j) * gradient(i)}.
  \item Interior flux \eqref{eq:interior}: built with \texttt{FEInterfaceValues} using \texttt{jump\_in\_values(i,q)} for the test and \texttt{average\_of\_values(j,q)} and \texttt{jump\_in\_values(j,q)} for the trial, scaled by $\beta=\bm b\!\cdot\!\bm n$ and \texttt{theta}.
  \item Boundary treatment \eqref{eq:bdry-matrix}--\eqref{eq:bdry-rhs}: inflow/outflow chosen by the sign of $\beta\,u_{\text{ext}}^n/2$; inflow uses exact $g$ at $t^{n+1}$.
  \item Time system \eqref{eq:time_system}: \texttt{system\_matrix\_time = (1/dt) M + A}, RHS \texttt{(1/dt) M u\^n + rhs}.
\end{itemize}

\end{document}
